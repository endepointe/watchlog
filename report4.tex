\documentclass{article}
\usepackage{geometry}
\usepackage{hyperref}
\usepackage{enumitem}

\geometry{a4paper, margin=1in}
\setlength{\parindent}{0pt}

\title{Progress Report\\[0.5em]
\large Watch-Log}
\author{Alvin Johns\\
CS463\\
Date: April 26, 2024}
\date{}

\begin{document}

\maketitle

\section*{Individual Tasks}
\subsection*{Completed (in the last seven days)}
This week had a major success. Using a simplified approach to reading multiple
file sources as spawned background tasks, a manager is now able to control the
watch-log binary using signals.

At first, I thought using the tokio library would be needed but apparently this
library provides little, if any, benefit when the handling multiple files. It was
recommended to use std lib threads and signal-hook.
\begin{itemize}
    \item Removed socket approach
    \item Using signals and threads
\end{itemize}

\subsection*{Planned (for the next progress report)}
\begin{itemize}
    \item Parse toml config file, setting the default sources on server startup.
        (easy, should be done by this weekend).
    \item Write appropriate tests and continue to move fast.
    \item Write documentation now that there is a working solution.
\end{itemize}

\subsection*{Links to Issues or Pull Requests}
\begin{itemize}
    \item main: \url{https://github.com/endepointe/watch-log}
\end{itemize}

\section*{Open Questions, Concerns, or Blockers}
\begin{itemize}
    \item Windows implementation has to be set off. This is a signals issue and
        will probably require a completely new codebase to run when windows is
        detected. 
\end{itemize}

\section*{Major Changes in Assets this Week}
\begin{itemize}
    \item Removed tokio.
    \item Not using sockets.
    \item Added a manager menu (client binary) that controls the watch-log with
        signals.
\end{itemize}

%\section*{Additional Documents TODO}
%\begin{itemize}[label={--}]
%    \item Product Requirements Document: \url{link}
%    \item Software Development Process: \url{link}
%    \item Software Design and Architecture: \url{link}
%    \item Product Design, User Persona, User Research, etc.: \url{link}
%    \item Documentation: \url{link}
%\end{itemize}

\end{document}

